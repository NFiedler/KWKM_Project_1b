% This is samplepaper.tex, a sample chapter demonstrating the
% LLNCS macro package for Springer Computer Science proceedings;
% Version 2.20 of 2017/10/04
%
\documentclass[runningheads]{llncs}
%
\usepackage{graphicx}
\usepackage[ngerman]{babel}
% Used for displaying a sample figure. If possible, figure files should
% be included in EPS format.
%
% If you use the hyperref package, please uncomment the following line
% to display URLs in blue roman font according to Springer's eBook style:
% \renewcommand\UrlFont{\color{blue}\rmfamily}

\begin{document}
%
\title{Kognitive Assistenzsysteme im Einsatz von hybriden Kollaborationsprozessen}
%
%\titlerunning{Abbreviated paper title}
% If the paper title is too long for the running head, you can set
% an abbreviated paper title here
%
\author{Niklas Fiedler\inst{1} \and
Jonas Hagge\inst{1} \and
Johannes Hedicke\inst{1} \and
Theresa Naß\inst{1} }
%
\authorrunning{Gruppe 1b}
% First names are abbreviated in the running head.
% If there are more than two authors, 'et al.' is used.
%
\institute{Universität Hamburg, Hamburg, Germany
\email{firstname.lastname@studium.uni-hamburg.de}}
%
\maketitle              % typeset the header of the contribution
%
\begin{abstract}
Wir präsentieren zwei Systeme die einem Moderator im Bürgerbeteiligungsworkshop zu der städtebaulichen Planung unterstützen können.

Einerseits tauchen in von den Bürgern geschriebenen Kommentaren häufig datenschutzrechtliche Informationen auf, die aufwendig von der Redaktion erst gefunden und entfernt werden müssen.
Deshalb präsentieren wir eine Methode die Namen in den Kommentaren finden kann und den Moderator informieren kann.

Andererseits werden sehr viele Kommentare geschrieben und die Moderatoren müssen alle lesen um einen Überblick über die Bürgermeinung zu bekommen. 
Um diesen Prozess zu erleichtern, stellen wir einen Ansatz vor der die Kommentare analysiert und häufige Schlagwortkombinationen findet.

\keywords{Kognitive Assistenzsysteme  \and Hybride Kollaborationsprozesse \and Städtebauplanung}
\end{abstract}
%
%
%
\section{Motivation}
\subsection{Bürgerbeteiligungsprozesse in der Städteplanung}
Da die Projekte die durch die Städteplanung entworfen werden relevant sind für alle Bürger in der Stadt, sollten diese auch einbezogen werden.
Einerseits kann dies zur Verbesserung der Planungen führen, da Bürger, die in der Region leben Vorschläge einbringen können die besonders wichtig sind für diese Region.
Wenn zum Beispiel ein Spielplatz gebaut werden soll, aber schon ein Spielplatz in der Nähe vorhanden ist, der bei der Planung übersehen wurde, kann man durch die Bürgereinbeziehung dies in die Planung mit einbeziehen.
Andererseits können so bereits früh mögliche Proteste der Bürger beachtet werden und man kann sich auf einen Kompromiss einigen mit dem sowohl die Bürger als auch die Planer zufrieden sind.
Dies kann viel Geld sparen, weil durch frühe Kompromisse der Planungsprozess noch nicht weiter fortgeschritten ist und noch nicht mit bauen begonnen wurde.

\subsection{Datenschutz}
Aus Datenschutzrechtlichen Gründen ist es nicht erlaubt, dass die Bürger persönlich identifizierende Informationen in diesen Prozess einfließen lassen.
Aus Unwissen der Bürger darüber passiert es aber regelmäßig, dass Kommentare mit vollem Namen unterschrieben werden.
Das führt dazu, dass die Moderatoren zeitintensiv und zeitnah alle Kommentare lesen müssen und ggf. identifizierende Informationen redaktionell entfernen müssen.
Dieser Prozess ist ineffizient, weil er aus sehr viel manuellem Aufwand besteht.
Dies führt zu weniger Arbeitskraft die in die tatsächliche Planung des Projektes investiert werden kann.

\subsection{Erkennung von relevanten Schlagworten}


\end{document}
